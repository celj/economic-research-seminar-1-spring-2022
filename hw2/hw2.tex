\documentclass[9pt,twocolumn,twoside,]{pnas-new}

% Use the lineno option to display guide line numbers if required.
% Note that the use of elements such as single-column equations
% may affect the guide line number alignment.


\usepackage[T1]{fontenc}
\usepackage[utf8]{inputenc}

% tightlist command for lists without linebreak
\providecommand{\tightlist}{%
  \setlength{\itemsep}{0pt}\setlength{\parskip}{0pt}}




\templatetype{pnasmathematics}  % Choose template

\title{Understanding Policy Change: How to Apply Political Economy
Concepts in Practice}

\author[]{Carlos Enrique Lezama Jacinto}

  \affil[]{Instituto Tecnológico Autónomo de México}


% Please give the surname of the lead author for the running footer
\leadauthor{}

% Please add here a significance statement to explain the relevance of your work
\significancestatement{}


\authorcontributions{}



\correspondingauthor{\textsuperscript{} One-pager written for the
research seminar on institutional economics. E-mail:
\href{mailto:clezamaj@itam.mx}{\nolinkurl{clezamaj@itam.mx}}.}

% Keywords are not mandatory, but authors are strongly encouraged to provide them. If provided, please include two to five keywords, separated by the pipe symbol, e.g:


\begin{abstract}

\end{abstract}

\dates{This manuscript was compiled on \today}
\doi{\url{www.pnas.org/cgi/doi/10.1073/pnas.XXXXXXXXXX}}

\begin{document}

% Optional adjustment to line up main text (after abstract) of first page with line numbers, when using both lineno and twocolumn options.
% You should only change this length when you've finalised the article contents.
\verticaladjustment{-2pt}



\maketitle
\thispagestyle{firststyle}
\ifthenelse{\boolean{shortarticle}}{\ifthenelse{\boolean{singlecolumn}}{\abscontentformatted}{\abscontent}}{}

% If your first paragraph (i.e. with the \dropcap) contains a list environment (quote, quotation, theorem, definition, enumerate, itemize...), the line after the list may have some extra indentation. If this is the case, add \parshape=0 to the end of the list environment.

\acknow{}

\hypertarget{research-question-tested-hypothesis-main-argument}{%
\section*{Research Question / Tested Hypothesis / Main
Argument}\label{research-question-tested-hypothesis-main-argument}}
\addcontentsline{toc}{section}{Research Question / Tested Hypothesis /
Main Argument}

Political economy: what it is and what it is not. Political economy
tries to explain multiple results in terms of economic development
through the identification of the agents' incentives and the context in
which they make decisions and interact strategically. It also considers
the government decisions and the economic-political environment to show
how they are responsible for some interactions between multiple economic
agents.

Political economy combines these two approaches and asks bidirectional
questions:

\begin{itemize}
\tightlist
\item
  How do political factors and institutions affect development outcomes?
\item
  How do economic factors and institutions related to development affect
  political outcomes?
\end{itemize}

\hypertarget{model-assumptions}{%
\section*{Model / Assumptions}\label{model-assumptions}}
\addcontentsline{toc}{section}{Model / Assumptions}

N.A.

\hypertarget{methodology}{%
\section*{Methodology}\label{methodology}}
\addcontentsline{toc}{section}{Methodology}

First, they choose to define what political economy is and what it is
not. Then, examples of how to understand the world through the lens of
political economy on important targets such as interest rates,
decentralization, and public financial management. Finally, they present
empirical constraints and limitations on political-economy analysis.

\hypertarget{data}{%
\section*{Data}\label{data}}
\addcontentsline{toc}{section}{Data}

Plots attempt to compare public spending versus outcomes in health and
education. Spending is measured as total annual public spending for
education on each primary-school-age child, in 1995 U.S. dollars,
averaged for two decades: the 1980s and the 1990s. For the health
graphs, the data show total annual per capita public spending on health,
in 1995 U.S. dollars, averaged for two decades: the 1980s and the 1990s.
The data was obtained from the World Bank database.

\hypertarget{data-inventory}{%
\section*{Data Inventory}\label{data-inventory}}
\addcontentsline{toc}{section}{Data Inventory}

Although the data may be downloaded from the World Bank database for the
analyzed countries and years, they do not show statistical rigor or
replicable methods.

\showmatmethods
\showacknow
\pnasbreak



% Bibliography
% \bibliography{pnas-sample}

\end{document}
