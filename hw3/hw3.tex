\documentclass[9pt,twocolumn,twoside,]{pnas-new}

% Use the lineno option to display guide line numbers if required.
% Note that the use of elements such as single-column equations
% may affect the guide line number alignment.


\usepackage[T1]{fontenc}
\usepackage[utf8]{inputenc}

% tightlist command for lists without linebreak
\providecommand{\tightlist}{%
  \setlength{\itemsep}{0pt}\setlength{\parskip}{0pt}}




\templatetype{pnasmathematics}  % Choose template

\title{Institutions, Institutional Change and Economic Performance}

\author[]{Carlos Enrique Lezama Jacinto}

  \affil[]{Instituto Tecnológico Autónomo de México}


% Please give the surname of the lead author for the running footer
\leadauthor{}

% Please add here a significance statement to explain the relevance of your work
\significancestatement{}


\authorcontributions{}



\correspondingauthor{\textsuperscript{} One-pager written for the
research seminar on institutional economics. E-mail:
\href{mailto:clezamaj@itam.mx}{\nolinkurl{clezamaj@itam.mx}}.}

% Keywords are not mandatory, but authors are strongly encouraged to provide them. If provided, please include two to five keywords, separated by the pipe symbol, e.g:


\begin{abstract}

\end{abstract}

\dates{This manuscript was compiled on \today}
\doi{\url{www.pnas.org/cgi/doi/10.1073/pnas.XXXXXXXXXX}}

\begin{document}

% Optional adjustment to line up main text (after abstract) of first page with line numbers, when using both lineno and twocolumn options.
% You should only change this length when you've finalised the article contents.
\verticaladjustment{-2pt}



\maketitle
\thispagestyle{firststyle}
\ifthenelse{\boolean{shortarticle}}{\ifthenelse{\boolean{singlecolumn}}{\abscontentformatted}{\abscontent}}{}

% If your first paragraph (i.e. with the \dropcap) contains a list environment (quote, quotation, theorem, definition, enumerate, itemize...), the line after the list may have some extra indentation. If this is the case, add \parshape=0 to the end of the list environment.

\acknow{}

\hypertarget{research-question-tested-hypothesis-main-argument}{%
\section*{Research Question / Tested Hypothesis / Main
Argument}\label{research-question-tested-hypothesis-main-argument}}
\addcontentsline{toc}{section}{Research Question / Tested Hypothesis /
Main Argument}

This part analyzes the nature of the institutions and how they affect
the performance of economies (or societies). Furthermore, it develops a
theory of institutional change to explain how the past influences the
present and the future through changes in institutions set at a time.
The main objective is to understand the different behaviors of economies
over time. The author affirms that institutions reduce uncertainty by
providing structure to people's lives, guiding their interactions.

\hypertarget{model-assumptions}{%
\section*{Model / Assumptions}\label{model-assumptions}}
\addcontentsline{toc}{section}{Model / Assumptions}

The model focuses on how individuals make decisions (microeconomics
approach). The author does not rely on traditional behavior assumptions
and searches for initial motivations on economic agents' decisions.
Therefore, our ``initial'' opportunity set plays a more relevant role
than scarcity. If economies realize the gains from trade by creating
relatively efficient institutions, it is because under certain
circumstances the private objectives of those with the bargaining
strength to alter institutions produce institutional solutions that turn
out to be or evolve into socially efficient ones.

\hypertarget{methodology}{%
\section*{Methodology}\label{methodology}}
\addcontentsline{toc}{section}{Methodology}

It focuses on Game Theory concepts such as \emph{cooperation} analyzing
transaction costs, restrictions, social roles, and duties.

\hypertarget{data}{%
\section*{Data}\label{data}}
\addcontentsline{toc}{section}{Data}

N.A.

\hypertarget{data-inventory}{%
\section*{Data Inventory}\label{data-inventory}}
\addcontentsline{toc}{section}{Data Inventory}

N.A.

\showmatmethods
\showacknow
\pnasbreak



% Bibliography
% \bibliography{pnas-sample}

\end{document}
